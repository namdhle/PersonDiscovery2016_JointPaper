\begin{abstract}

We describe the “Multimodal Person Discovery in Broadcast TV” task of MediaEval 2015 benchmarking initiative. Participants were asked to return the names of people who can be both seen as well as heard in every shot of a collection of videos. The list of people was not known a priori and their
names had to be discovered in an unsupervised way from media content using text overlay or speech transcripts. The task was evaluated using information retrieval metrics, based on a
posteriori collaborative annotation of the test corpus. The first edition of the task gathered 9 teams which submitted 34 runs. This paper provides quantitative and qualitative comparisons of
participants submissions. We also investigate why all systems failed for particular shots, paving the way for future promising research directions.

\end{abstract}

\endinput
