\begin{abstract}
The rapid growth of multimedia databases and the human interest in their peers make 
indices representing the location and identity of people in audio-visual documents essential for searching archives.
%human nature leads people to be very interested in other people.
%
Person discovery in the absence of prior identity knowledge requires accurate association of audio-visual cues and detected names. 
%
To this end, we present 3 different strategies to approach this problem: clustering-based naming, verification-based naming, and graph-based naming.
%
Each of these strategies utilizes different recent advances in unsupervised face / speech representation, verification, and optimization.
%
To have a better understanding of the approaches, this paper also provides a quantitative and qualitative comparative study of these approaches using the associated corpus of the Person Discovery challenge at MediaEval 2016.
%
From the results of our experiments, we can observe the pros and cons of each approach, thus paving the way for future promising research directions.
%
%To facilitate a comprehensive study, we conduct the experiments on the associated corpus of the Person Discovery challenge at MediaEval 2016.

\end{abstract}

\endinput
