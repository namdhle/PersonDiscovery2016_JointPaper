\section{Future works}
\label{sec:discuss}

The organization of a new task on multimodal person discovery at MediaEval 2015 benchmarking initiative was well received by the community, with 9 participating teams interested in the idea of unsupervised person identification.
The provision of a modular baseline system lowered the entry gap for the competing teams with varying scientific backgrounds. The live leaderboard allowed to overcome the condition mismatch between the development and the test
set. The evaluation workflow relied on the CAMOMILE framework with a posteriori annotation of the corpus by the
organizers and the participants. Very good results could thus be achieved with 82.6\% MAP for the best system. However,
the analysis of the results shows that a simple strategy of propagating the written names to the speaker diarization output
was sufficient, mainly due to the nature of the test corpus. This leads us to several directions for future editions of the
task. First, we should improve the video quality and increase the content variety of the corpus; the respective weight of face
tracking and speaker diarization would be better balanced, and the detection of pronounced names would become more useful.
Also, the queries may be restricted to persons occuring in at least two shows, making the task definitively harder but also reducing the annotation need.

\endinput
