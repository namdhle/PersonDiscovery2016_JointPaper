\section{Future works}
\label{sec:discuss}

We have presented three different methodologies to perform unsupervised person identification in broadcast news. The quantitative analysis was done on the associated corpus of the Multimodal Person Discovery challenge of MediaEval 2016. In this challenge, person discovery is benchmarked as an index retrieval problem, in which indices represent shots when a person appears and speaks.
%
%
From the experiments, we can observe that clustering-based methods still achieve better accuracy than the alternatives. The results also suggest potential directions to improve verification-based and graph-based methods
%
by increasing the quality of OCR, hyper parameter tuning, or discriminative talking face detection.
%
%Combining with clustering-based methods to improve the robustness of the discriminative models, talking face detection module.
%
On the other hand, these two approaches have many interesting improvements such as discriminative models or unified audio-visual similarity, which can be exploited by combining them with clustering-based methods.
%
Our results also emphasize the importance of multimodal processing, which is a future direction of our work.


\endinput
