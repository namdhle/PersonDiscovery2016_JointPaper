\section{Introduction}

As the retrieval of information on people in videos is of high interest for users, algorithms indexing identities of people and retrieving their respective quotations are indispensable for searching archives. This practical need leads to research problems on how to index people presence in videos.
%
%TV archives maintained by national institutions such as the French INA, the Netherlands Institute for Sound \& Vision, or the BBC are rapidly growing in size. The need for applications that make these archives searchable has led researchers to devote concerted effort to developing technologies that create indexes.
%
%Because human nature leads people to be very interested in other people.
%Indexes that represent the location and identity of people in the archive are indispensable for searching archives.
%
Started in 2011, the REPERE challenge aimed at supporting research on multimodal person recognition~\cite{BERNARD--SLAM--2013, GIRAUDEL--LREC--2012}. Its main goal was to answer the two questions \emph{``who speaks when?''} and \emph{``who appears when?''} using any available source of information (including pre-existing biometric models and person names extracted from the videos).
%from text overlay and speech transcripts). 
%
Thanks to this challenge and the associated multimodal corpus~\cite{GIRAUDEL--LREC--2012}, significant progress was achieved in either supervised or unsupervised multimodal person recognition~\cite{BECHET--INTERSPEECH--2014, BREDIN--IJMIR--2014, GAY--CBMI--2014, poignant2012fusion, ROUVIER--CBMI--2014}.

However, when the content is created or broadcast, it is not always possible to predict which people will be the most important to find in the future and biometric models may not yet be available at indexing time.
%
The goal of this task is thus to address the challenge of indexing people in the archive under real-world conditions, \emph{i.e.} when there is no pre-set list of people to index.
%
This makes the task completely unsupervised.
%
In order to successfully tag people with the correct identities, names must first be detected from audio-visual sources such as automatic transcripts (ASR) or optical character recognition (OCR).
%
Then one must find a way to assign a name correctly to a presence of the corresponding person, and that name must also be propagated to all the shots during which that person appears and speaks. 
%
A standard approach to solve this is first based on face / speech clustering to partition a videos into homogeneous segments according to identities, followed by the assignment of names to segments appropriately.
%
Although commonly used in state-of-the-art systems~\cite{le2015eumssi,poignant2012fusion}, it suffers from several drawbacks such as potential errors of face / speech clustering or the lack of straightforward way to combine audio-visual streams.
%
In order to alleviate these drawbacks of clustering-based naming %as well as to take advantage of recent advances in verification and 
two alternative strategies are proposed based on verification and graph optimization. 
%
All these three strategies share some common building blocks such as face / speech representation, person diarization, or audio-visual verification. Though each of these blocks has been well studied within its respective context~\cite{parkhi15deep,wallace2012total,Schroff2015,Ben}. 
%
However, these approaches have never been fully investigated and compared as whole systems in the large-scale multimedia indexing context before. Thus in this paper, the authors aim to investigate all these approaches with variations in their components using real world datasets from TV news. 
%
To emphasize on the unsupervised nature of real world applications, we applied these approaches on a large scale multimedia dataset associated to the ``Multimodal Person Discovery in Broadcast TV'' task~\cite{POIGNANT--MEDIAEVAL--2015,bredin2016mediaeval}.
%
The benchmarking results allow us to analyse all 3 approaches to understand their pros and cons to draw lessons for good practice in large-scale person discovery in broadcast news.

The next Section introduces more details into the Person Discovery challenge about its corpus and evaluation protocol.
Then Section \ref{sec:overview} gives an overview about our approaches while Section \ref{sec:clustering},~\ref{sec:verification}, and~\ref{sec:graph} describe the methodologies in full details.
Section \ref{sec:experiment} presents the conducted experiments and analysis.
%
Finally, Section \ref{sec:discuss} concludes the paper 
with further discussion and future works.

\endinput
